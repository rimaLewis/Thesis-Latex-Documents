\ifgerman{\chapter{Grundlagen}}{\chapter{Background}}
\label{background}

%- Allgemeine Wissensgrundlagen des Fachgebiets
%- Spezielle Grundlagen, die für das Verständnis erforderlich sind
%- Rahmenbedingungen für die Arbeit
%- Ausführungen zum Stand des Wissens / der Technik
%Als Leitprinzip gilt: Nur Informationen erwähnen, die
%- später benötigt werden,
%- notwendig sind, um die Arbeit oder ihre Motivation zu verstehen
%Das heißt insbesondere,
%- keine Inhalte aus Lehrbüchern, außer
%- diese werden benötigt, um Problemstellung oder Lösungsweg zu definieren.

% \gls{IDE}
 
\todots{

  - Explain what is 
	                -> configurable software systems
	                -> Performance-Influence Models
	                -> Configuration options
	                -> Interactions
	                -> Positive and negative influences
	            - Research Questions
	            - Visualization techniques 
	            - Things explained in the warm-up phase of the interview
	            - No need to explain Feature Models
	            - Use one example of Preformance-Influence Model throughout the documentation
	            - Example should include configuration options, Interactions, positive terms, negative terms-}