\ifgerman{\chapter{Grundlagen}}{\chapter{Background}}
\label{background}

%- Allgemeine Wissensgrundlagen des Fachgebiets
%- Spezielle Grundlagen, die für das Verständnis erforderlich sind
%- Rahmenbedingungen für die Arbeit
%- Ausführungen zum Stand des Wissens / der Technik
%Als Leitprinzip gilt: Nur Informationen erwähnen, die
%- später benötigt werden,
%- notwendig sind, um die Arbeit oder ihre Motivation zu verstehen
%Das heißt insbesondere,
%- keine Inhalte aus Lehrbüchern, außer
%- diese werden benötigt, um Problemstellung oder Lösungsweg zu definieren.

% \gls{IDE}
 
A software or a system with different configuration options is called configurable software system. A configuration option adds a definite functionality to the system, it can be optional or mandatory. A configuration option can be selected or deselected. An example for Configuration option can be compression and encryption.
when several configuration options are selected at the same time, they may interact with each other, when they do it is called an Interaction.
\break
\newline
Often times an user or a developer is interested in knowing how the configuration options and/or interactions have an impact over the performance of the system. To know the performance influence we use a tool called 'SPL Conqueror'.
SPL Conqueror takes as an input a Configurable software system, and produces a set of valid Performance-Influence models as an output.

\textbf{An example for Performance-Influence Model: }

\begin{equation*}
  \pi {(c)} = \overbrace{\underbrace {3}_{Coeff.} \cdot  \underbrace{{c(A)}}_{Option}}^{Term 1}  + \overbrace{ \underbrace{5}_{Coeff.} \cdot \underbrace{{c(B)}}_{Option}}^{Term 2} + \overbrace{\underbrace{0}_{Coeff.} \cdot \underbrace{{c(C)}}_{Option}}^{Term 3} - \overbrace{\underbrace{4}_{Coeff.} \cdot \underbrace{{ c(A)} \cdot {c(B)}}_{Interaction}}^{Term 4}
\end{equation*}



Every performance-influence model consists of a set of terms. Each term has the participating configuration option/interactions and a co-efficient. This co-efficient indicates the performance that the configuration option/interaction contributes towards the total performance of the system.
Performance here is calculated in terms of its execution time. A positive value indicates that the performance of the system decreases, whereas a negative term indicates that it increases the performance of the system. In the above given example there are four terms, the first term has the participating configuration option 'A', and it contributes 3 units towards the total performance of the system, Term 2 has the participating option 'B' and it contributes 5 units towards the total performance of the system, similarly term 3 contributes 0 units, which says that it does not have an influence towards the performance of the system when configuration option 'C' is selected. The fourth term has two configuration option 'A' and 'B', which when selected together interact with each other, hence it is called an Interaction, and it contributes 4 units towards the total performance of the system. The negative sign before the co-efficient indicates that it has increases the performance of the system. while the first 3 terms decreases the performance of the system.
\newline
These Performance-Influence models can be visualized, which makes it easier to interpret them. Once we have the visualization we can identify the most relevant configuration option/interaction.
A software may have an upgraded version, in this case we can compare the current and the upgraded versions of the software to identify the differences in the performance influence the configuration options/interactions may have.
As and when the software is upgraded to add new configuration options, the size of the visualization increases to reflect the newly added configuration options. The visualization scales as the software keeps adding or removing configuration options.
                                                                       


	