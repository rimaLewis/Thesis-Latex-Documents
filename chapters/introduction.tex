\ifgerman{\chapter{Einführung}}{\chapter{Introduction}}
%- Hintergrund
%- Motivation
%- Ziele
%- Aufgaben
%- Allgemeine Beschreibung des Projektes
%- Worum geht es in dieser Arbeit?
%- Wer hat die Arbeit veranlasst und wozu?
%- Wer soll von den Ergebnissen profitieren?
%- Welches Problem soll gelöst werden? Warum?
%- Unter welchen Umständen braucht man eine Verbesserung?
%- Was ist der Stand der Technik?
%- Welche noch offenen Probleme gibt es?
%- Worin unterscheidet sich mein Ansatz von den bisherigen?
%- Welche Ziele hat die Arbeit?
%- Wie will ich diese Ziele erreichen?
%- Was habe ich im Einzelnen vor? 

\todots

\section{Goal of this Thesis}
Performance-Influence Models helps understanding how performance of a software system is affected, but when a performance-influence model is very large, it is not easily readable. The complexity of the model can increase when the configuration features of a software system increases.
The model in its original format is not easy to draw conclusions.
As a solution to this problem, we use these models to map into several visualizations, based on which the user/developer can draw conclusions easily.
The motivation behind the thesis is to understand large performance-influence models easily. which cannot be done in its original text format.
Visualization of any data makes it easier to understand it, for example to know the outlier in a large set of data.

Most of the software systems are configurable to the user. The  software systems when configured affects certain functional and non-functional requirements of the software. In this thesis we focus on non-functional requirements, for example Performance. These software systems should be optimized in a way to affect performance positively.

\todots{

- What are the problems with understanding performance-influence models?
				- what are the possible solutions to the problems?
				- Motivation behind the thesis
				- why do we need to visualize the Performance-Influence Models?	
				- Things explained in the slides (first presentation)	}

\ifgerman{\section{Gliederung der Arbeit}}{\section{Structure of the Thesis}}

\todots 
