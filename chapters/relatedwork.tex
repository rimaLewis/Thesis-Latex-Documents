\ifgerman{\chapter{Verwandte Arbeiten}}{\chapter{Related Work}}
\label{relatedwork}

Visualization specifically based on performance-influence models has not been an area of research up until  now, although we have studies related to performance visualization techniques in recent past that helps developers and analysts in improving the time and energy efficiency of the software. The paper regarding the state of the art performance visualization by Isaacs \textit{et al}. \cite{DBLP:conf/vissym/IsaacsGJGB0HB14},  mainly focuses on the current state of the art of performance visualization. This paper surveys existing work on performance visualization, and categorize the goals that these performance visualization techniques can answer. The results from the survey are organized into areas depending on which the visualization is constructed and describe the state of art research for each area. However, due to the fact that there are less number of domain experts, in fields like High Performance Computing, the usability study of performance visualization techniques was not done on the surveyed papers. 

Another research on performance analysis on cluster performance data by Haynes \textit{et al}. \cite{DBLP:conf/cluster/HaynesCR01}, presents a 3-D visualization tool to visually represent the performance data from a large scale cluster for analyzing. The visualization displays data in the context of complex cluster interconnection topologies. It aids analysts to discover the cause of issues ranging from communication bottlenecks to hardware errors. The effectiveness of this visualization tool is performed on clusters on two separate instances. The first instance of usability  shows that using visualization can minimize the time taken to diagnose hardware problems in a large system. The second instance of usability demonstrates that visualization can provide insight for understanding system and job performance.

A similar research on scalability by M{\"{u}}ller \textit{et al}. \cite{DBLP:conf/parco/MullerKJLBMN07}, presents scalability studies performance analysis tool Vampir and VampirTrace. The anaylsis of Vampir tools family is done on real applications taken from SpecMPI benchmark suite. Vampir is a well known performance analysis framework. Vampir and VampirServer provide an interactive visualization of dynamic program behaviour. They depend on loading the trace data to main memory completely to begin the performance analysis session. The evaluation with VampirServer looks quite promising, provided enough distributed memory. Hence, the software architecture is suitable for distributed memory platforms.

A similar work done by Bhatele \textit{et al}. \cite{DBLP:conf/sc/BhateleGIGSBH12}, aims at visualizing performance data of large scale adaptive applications. It presents a scalable visualization technique that combines hardware and communication data providing an extensive diagnosis of a scalability problem that caused a bottleneck in the AMR library. The evaluation showed that the mitigation strategy improves the performance of the AMR library by 22\% for a 65,536 core run on a Blue Gene/P system. 
