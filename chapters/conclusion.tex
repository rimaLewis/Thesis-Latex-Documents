\ifgerman{\chapter{Zusammenfassung}}{\chapter{Conclusion}}
\label{conclusion}

Performance-influence models are used to interpret the influence of configuration options like encryption on non-functional properties like the performance of a configurable software system. With the increasing complexity of performance-influence models, it is difficult to interpret them in their original representation. Therefore, we presented different visualization techniques to ease the interpretation of performance-influence models.

We selected 3 visualization techniques; the radar plot, the text plot, and the ratio plot. To assess the quality of these visualization techniques we conducted an interview. 

Research questions were selected based on the number of performance-influence models the visualizations included. The visualizations could perform differently regarding the number of performance-influence models and, thus, we investigate how they perform with one performance-influence model, two performance-influence models, and many performance-influence models.

From the interview and their results, we found out that the text plot outperformed the radar plot and the ratio plot. The text plot presents the performance influence in a vertically aligned format and hence it is easier to perceive data than on the radar and the ratio plot. 



