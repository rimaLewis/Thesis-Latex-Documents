\ifgerman{\chapter{Zusammenfassung}}{\chapter{Conclusion}}
\label{conclusion}

Performance-influence models are used to interpret the influence of functional properties like encryption on non-functional properties like performance of a configurable software system. With their increasing complexity of performance-influence models, it is difficult to interpret them in their original representation. Therefore, we presented different visualization techniques to ease the interpretation of performance-influence models.

We selected 3 visualization techniques; radar plot, text plot and ratio plot. To assess the quality of these visualization techniques we conducted an interview. 

Research questions were selected based on the number of performance-influence models the visualizations included. For this thesis, we selected research questions based on one performance-influence model, two performance-influence models and many performance-influence models.

From the interview and their results, we found out that text plot outperformed radar plot and ratio plot. Text plot presents the performance influence in a textual format and hence it is easier to perceive data than on radar and ratio plot.

Ratio plot performed better on many performance-influence, since comparison on ratio plot are color coded for each configuration option or interaction, than in radar and text plot where only each performance-influence model is presented with a different color.

In general, text plot performed better in terms of number of correct answers it produced and the time taken to interpret this visualization technique.




