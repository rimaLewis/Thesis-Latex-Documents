\usepackage{booktabs}
\newcommand{\ra}[1]{\renewcommand{\arraystretch}{#1}}
\usepackage{pdfpages}
\usepackage{graphicx}
\usepackage{booktabs}
\raggedbottom
\usepackage{forest}
\usepackage{relsize}
\usepackage{breqn}
\usepackage[font=small,labelfont=bf]{caption}
\usepackage[bookmarksnumbered=true]{hyperref}
\hypersetup{
     colorlinks = true,
     linkcolor = blue,
     anchorcolor = blue,
     citecolor = blue,
     filecolor = blue,
     urlcolor = blue
     }
\usepackage{pgfplotstable}


\definecolor{folderbg}{RGB}{124,166,198}
\definecolor{folderborder}{RGB}{110,144,169}

\def\Size{4pt}
\tikzset{
  folder/.pic={
    \filldraw[draw=folderborder,top color=folderbg!50,bottom color=folderbg]
      (-1.05*\Size,0.2\Size+5pt) rectangle ++(.75*\Size,-0.2\Size-5pt);  
    \filldraw[draw=folderborder,top color=folderbg!50,bottom color=folderbg]
      (-1.15*\Size,-\Size) rectangle (1.15*\Size,\Size);
  }
}


\usepackage{pgfplots}
    \pgfplotscreateplotcyclelist{my colors}{
        % color for the legend
        black!50\\
        % color for the "real" plots
        green\\
        orange\\
        blue\\
        red\\
        black\\
        cyan\\
    }
    \pgfplotsset{
        compat=1.3,
        cycle list name=my colors,
        legend cell align=left,
    }
\usepackage{caption}
\usepgfplotslibrary{statistics}
\usepackage{multirow}
\usepackage{float}
\usepackage{amssymb}
\usepackage{adjustbox}
\usepackage{mdframed}
\usepackage{lipsum}% example text
\usepackage{array,booktabs}
\newcolumntype{M}[1]{>{\centering\arraybackslash}m{#1}}
\mdfsetup{
    linewidth=0.6pt
}
\makeatletter
\DeclareMathOperator*{\sometext}{text} % "
\newenvironment{myindentpar}[1]%
  {\begin{list}{}%
          {\setlength{\leftmargin}{#1}}%
          \item[]%
  }
  {\end{list}}
\usepackage{enumitem}
\usetikzlibrary{matrix}
\usepgfplotslibrary{groupplots}
\pgfplotsset{compat=newest}

\usetikzlibrary{matrix}
\usepgfplotslibrary{groupplots}
\pgfplotsset{compat=newest}

\newcommand\varpm{\mathbin{\vcenter{\hbox{%
  \oalign{\hfil$\scriptstyle+$\hfil\cr
          \noalign{\kern-.3ex}
          $\scriptscriptstyle{-}$\cr}%
}}}}

% we use \prefix@<level> only if it is defined
\renewcommand{\@seccntformat}[1]{%
  \ifcsname prefix@#1\endcsname
    \csname prefix@#1\endcsname
  \else
    \csname the#1\endcsname\quad
  \fi}
% define \prefix@section
\newcommand\prefix@section{Section \thesection: }
\makeatother

\newlength{\ridiculouslylargecmidrulesep}
\setlength{\ridiculouslylargecmidrulesep}{50pt}
\usepackage{pgfplots}
\usepackage{tikz}
	\usetikzlibrary{arrows,positioning,backgrounds,fit,trees} 
	\usetikzlibrary{fadings,shapes.geometric}
	\usetikzlibrary{decorations,scopes,calc,decorations.pathreplacing}

% Tortendiagramme
\newcommand{\slice}[4]{
  \pgfmathparse{0.5*#1+0.5*#2}
  \let\midangle\pgfmathresult

  % slice
  \draw[thick,
	%fill=background
	] (0,0) -- (#1:1) arc (#1:#2:1) -- cycle;

  % outer label
  \node[label=\midangle:#4] at (\midangle:1) {};

  % inner label
  \pgfmathparse{min((#2-#1-10)/110*(-0.3),0)}
  \let\temp\pgfmathresult
  \pgfmathparse{max(\temp,-0.5) + 0.8}
  \let\innerpos\pgfmathresult
  \node at (\midangle:\innerpos) {#3};
}
\newcommand{\mypiechart}[2]{
	\begin{tikzpicture}[scale=#1]
		\newcounter{a}
		\newcounter{b}
		\foreach \p/\t in {#2}
			{
				\setcounter{a}{\value{b}}
				\addtocounter{b}{\p}
				\slice{\thea/100*360}
							{\theb/100*360}
							{\p\%}{\t}
			}
	\end{tikzpicture}
}
